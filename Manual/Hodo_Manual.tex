\documentclass[letterpaper,12pt]{article}
%documentclass[superscriptaddress,preprintnumbers,amsmath,amssymb,aps,11pt]{revtex4}
%\usepackage[]{authblk}
%\usepackage{graphics}
\usepackage[dvipdf]{graphics}
%\usepackage{subfig}  % For subfloats
\usepackage{color}
\usepackage[usenames,dvipsnames]{xcolor}
\usepackage{epsfig}
\usepackage{wrapfig}
\usepackage{rotating}
\usepackage{caption}
%\usepackage{subcaption}
\usepackage{subfig}
\usepackage{authblk}
\usepackage{hyperref}
\usepackage{url}
\usepackage{lineno}
%\linenumbers

\oddsidemargin = -14mm
\topmargin = -2.9cm
\textwidth = 19cm
\textheight = 24cm

\def \rarr {\rightarrow}
\def \grinp {\includegraphics}
\def \tw {\textwidth}
\def\dfrac#1#2{\displaystyle{{#1}\over{#2}}}
\def \dstl {\displaystyle}
\definecolor{GREEN}{rgb}{0.,0.8,0}
\definecolor{RED}{rgb}{1,0,0}
\definecolor{ORANGE}{rgb}{1,0.5,0}

\title{Manual for HPS Hodoscope}
\author{Rafayel Paremuzyan}

\begin{document}
 \maketitle
 
 \section{General description of the Hodoscope}
 
 The Hodoscope (hodo) is a charge particle detector. It's main purpose is to help to suppress large photon background at the trigger level. Main trigger of HPS will be a coincidence (latheral and time) between hits in Hodoscope and clusters in ECal. The hodo is installed insde the SVT vacuum chamber and is located in between the Layer 6 of HPS SVT and the ECal front face.
 %%%%%%%%%%%%%%%%%%%%%%%%%%%%%%%%%%%%% F I G U R E %%%%%%%%%%%%%%%%%%%%%%%%%%%%%%%%%%%%%
 \begin{figure}[!htb]
 \centering
  \grinp[width=0.95\tw]{img/HodoGeneral_Vewi1.pdf}
  \caption{An engineering drawing of the hodoscope. The pictur shows consize description of some parts.}
  \label{fig:hodo_gen_view1}
 \end{figure}
%%%%%%%%%%%%%%%%%%%%%%%%%%%%%%%%%%%%% F I G U R E %%%%%%%%%%%%%%%%%%%%%%%%%%%%%%%%%%%%%
The Fig.\ref{fig:hodo_gen_view1} shows the General view of the hodoscope.
 
 
 \section{Channel Mapping}
 \section{HV settings}
 \section{PMTs}
 \section{PMT housing}
 
\end{document}
